\documentclass[conference]{IEEEtran}
\usepackage[utf8]{inputenc}
\usepackage[main=english,portuguese]{babel}

\usepackage{color,soul} % TEMPORARIO => SÓ PARA VISUALIZAR TEXTO PENDENTE

\ifCLASSINFOpdf
\else
\fi
\hyphenation{op-tical net-works semi-conduc-tor}


\begin{document}

\title{Sistema de visualização de\\redes metabólicas em grafo}

\author{\IEEEauthorblockN{Gabriella de O. Esteves}
\IEEEauthorblockA{Universidade de Brasília\\
Departamento de Ciência da Computação\\
Brasília, Brasil\\
Email: gabepk.ape@gmail.com}}


\maketitle

% As a general rule, do not put math, special symbols or citations
% in the abstract
\begin{abstract}
The abstract goes here.
\end{abstract}

% no keywords

\section{Introduction}

O metabolismo é isso X, ocorre por isso (síntese e degradação) e funciona com isso (moléculas, metabólitos).

Como o metabolismo tem sido representado computacionalmente (redes metabólicas)? Como as redes metabólicas tem sido visualizadas (estado da arte)? \\
Problema, objetivo. \\
Descrição dos capítulos.


\section{Redes Metabólicas}

\hl{FRASE INTRO}. As reações bioquímicas são alterações químicas que fornecem um ou mais produtos a partir de uma ou mais entradas, chamadas de substratos. Uma via metabólica é uma sequência de reações bioquímicas, cujo produto e subtrato são denominados de metabólitos, que podem ser catalisadas por enzimas, estas que muitas vezes necessitam de compostos químicos não-proteicos chamados de co-fatores para realizarem suas atividades na célula. O conjunto de vias metabólicas de um organismo é chamado de rede metabólica. Todos estes elementos que compõem as redes metabólicas são dados biológicos estudados na área metabolômica. Nesta seção serão apresentados três bancos de dados de redes metabólicas utilizadas em análise do metaboloma, KEGG, BIoCyc e Reactoma, bem como suas ferramentas de visualização \hl{em grafo}.

\subsection{Conceitos de Biologia Molecular}

O DNA é um conjunto de biomoléculas em um organismo que armazenam informações, chamados de genes, referentes ao funcionamento de todas as suas células. Ele constitui o genoma em todos os seres vivos, com excessão dos vírus. A expressão dos genes é o processo no qual os genes são filtrados e utilizados na síntese de um produto, geralmente proteína. O método é segmentado em três etapas: transcrição (síntese de RNA mensageiro a partir de DNA), \textit{splicing} (filtragem do RNA mensageiro) e tradução (síntese de proteína a partir do RNA mensageiro filtrado). Completo este processo, as proteínas resultantes poderão formar uma configuração tridimensional de até quatro níveis. As enzimas, por exemplo, são proteínas que podem ter estrutura terciária ou quaternária.

\subsection{Conceitos de Metabolismo}


Reações bioquímicas
Metabolismo primário. Metabolismo secundário. Biosíntese/degradação. 
Atividade enzimática (isoenzima).

\subsection{Banco de Dados de redes metabólicas}

KEGG. BioCyc. Reactoma.


\section{Ferramentas de visualização de redes metabólicas}

Ferramenta do KEGG. Ferramentas do BioCyc. Ferramentas do Reactome browser.
Ferramentas do Cytoscape.

\section{Sistema 2Path}

\subsection{Banco de dados em grafo}

Waldeyr

\subsection{Sistema de consulta}

Gabriella

\section{Conclusão}

Conclusão





% An example of a floating figure using the graphicx package.
% Note that \label must occur AFTER (or within) \caption.
% For figures, \caption should occur after the \includegraphics.
% Note that IEEEtran v1.7 and later has special internal code that
% is designed to preserve the operation of \label within \caption
% even when the captionsoff option is in effect. However, because
% of issues like this, it may be the safest practice to put all your
% \label just after \caption rather than within \caption{}.
%
% Reminder: the "draftcls" or "draftclsnofoot", not "draft", class
% option should be used if it is desired that the figures are to be
% displayed while in draft mode.
%
%\begin{figure}[!t]
%\centering
%\includegraphics[width=2.5in]{myfigure}
% where an .eps filename suffix will be assumed under latex, 
% and a .pdf suffix will be assumed for pdflatex; or what has been declared
% via \DeclareGraphicsExtensions.
%\caption{Simulation results for the network.}
%\label{fig_sim}
%\end{figure}

% Note that the IEEE typically puts floats only at the top, even when this
% results in a large percentage of a column being occupied by floats.


% An example of a double column floating figure using two subfigures.
% (The subfig.sty package must be loaded for this to work.)
% The subfigure \label commands are set within each subfloat command,
% and the \label for the overall figure must come after \caption.
% \hfil is used as a separator to get equal spacing.
% Watch out that the combined width of all the subfigures on a 
% line do not exceed the text width or a line break will occur.
%
%\begin{figure*}[!t]
%\centering
%\subfloat[Case I]{\includegraphics[width=2.5in]{box}%
%\label{fig_first_case}}
%\hfil
%\subfloat[Case II]{\includegraphics[width=2.5in]{box}%
%\label{fig_second_case}}
%\caption{Simulation results for the network.}
%\label{fig_sim}
%\end{figure*}
%
% Note that often IEEE papers with subfigures do not employ subfigure
% captions (using the optional argument to \subfloat[]), but instead will
% reference/describe all of them (a), (b), etc., within the main caption.
% Be aware that for subfig.sty to generate the (a), (b), etc., subfigure
% labels, the optional argument to \subfloat must be present. If a
% subcaption is not desired, just leave its contents blank,
% e.g., \subfloat[].


% An example of a floating table. Note that, for IEEE style tables, the
% \caption command should come BEFORE the table and, given that table
% captions serve much like titles, are usually capitalized except for words
% such as a, an, and, as, at, but, by, for, in, nor, of, on, or, the, to
% and up, which are usually not capitalized unless they are the first or
% last word of the caption. Table text will default to \footnotesize as
% the IEEE normally uses this smaller font for tables.
% The \label must come after \caption as always.
%
%\begin{table}[!t]
%% increase table row spacing, adjust to taste
%\renewcommand{\arraystretch}{1.3}
% if using array.sty, it might be a good idea to tweak the value of
% \extrarowheight as needed to properly center the text within the cells
%\caption{An Example of a Table}
%\label{table_example}
%\centering
%% Some packages, such as MDW tools, offer better commands for making tables
%% than the plain LaTeX2e tabular which is used here.
%\begin{tabular}{|c||c|}
%\hline
%One & Two\\
%\hline
%Three & Four\\
%\hline
%\end{tabular}
%\end{table}


% Note that the IEEE does not put floats in the very first column
% - or typically anywhere on the first page for that matter. Also,
% in-text middle ("here") positioning is typically not used, but it
% is allowed and encouraged for Computer Society conferences (but
% not Computer Society journals). Most IEEE journals/conferences use
% top floats exclusively. 
% Note that, LaTeX2e, unlike IEEE journals/conferences, places
% footnotes above bottom floats. This can be corrected via the
% \fnbelowfloat command of the stfloats package.



% conference papers do not normally have an appendix


% use section* for acknowledgment
\section*{Agradecimento}


The authors would like to thank...





% trigger a \newpage just before the given reference
% number - used to balance the columns on the last page
% adjust value as needed - may need to be readjusted if
% the document is modified later
%\IEEEtriggeratref{8}
% The "triggered" command can be changed if desired:
%\IEEEtriggercmd{\enlargethispage{-5in}}

% references section
% manually copy in the resultant .bbl file
% set second argument of \begin to the number of references
% (used to reserve space for the reference number labels box)
\begin{thebibliography}{1}

\bibitem{IEEEhowto:kopka}
H.~Kopka and P.~W. Daly, \emph{A Guide to \LaTeX}, 3rd~ed.\hskip 1em plus
  0.5em minus 0.4em\relax Harlow, England: Addison-Wesley, 1999.

\end{thebibliography}





\end{document}


