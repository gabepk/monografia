%%%%%%%%%%%%%%%%%%%%%%%%%%%%%%%%%%%%%%%%
% Classe do documento
%%%%%%%%%%%%%%%%%%%%%%%%%%%%%%%%%%%%%%%%

% Nós usamos a classe "unb-cic".  Deixe apenas uma das linhas
% abaixo não-comentada, dependendo se você for do bacharelado ou
% da licenciatura.

\documentclass[bacharelado]{unb-cic}
%\documentclass[licenciatura]{unb-cic}



%%%%%%%%%%%%%%%%%%%%%%%%%%%%%%%%%%%%%%%%
% Pacotes importados
%%%%%%%%%%%%%%%%%%%%%%%%%%%%%%%%%%%%%%%%

\usepackage[brazil,american]{babel}
\usepackage[T1]{fontenc}
\usepackage{indentfirst}
\usepackage{natbib}
\usepackage[table,xcdraw]{xcolor}
\usepackage{color,soul}
\usepackage[utf8]{inputenc}
\DeclareUnicodeCharacter{2010}{-}% support older LaTeX versions
\usepackage{caption}
\usepackage{subfig}
\usepackage{graphicx}
\usepackage{multirow}
\usepackage{geometry}
\usepackage{listings}
\usepackage{pdfpages}

\usepackage{tikz}

\usetikzlibrary{arrows, calc, decorations.markings, positioning}
\setcounter{secnumdepth}{4}
\graphicspath{{./figuras/}}

% Para pular linha em células de tabela
\newcommand{\specialcell}[2][l]{%
  \begin{tabular}[#1]{@{}l@{}}#2\end{tabular}}

\pagestyle{empty}
\makeatletter
\newenvironment{timeline}[6]{%
    \newcommand{\startyear}{#1}
    \newcommand{\tlendyear}{#2}

    \newcommand{\yearcolumnwidth}{#3}
    \newcommand{\rulecolumnwidth}{#4}
    \newcommand{\entrycolumnwidth}{#5}
    \newcommand{\timelineheight}{#6}

    \newcommand{\templength}{}
    \newcommand{\entrycounter}{0}

    \long\def\ifnodedefined##1##2##3{%
        \@ifundefined{pgf@sh@ns@##1}{##3}{##2}%
    }
    \newcommand{\ifnodeundefined}[2]{%
        \ifnodedefined{##1}{}{##2}
    }

    \newcommand{\drawtimeline}{%
        \draw[timelinerule] (\yearcolumnwidth+5pt, 0pt) -- (\yearcolumnwidth+5pt, -\timelineheight);
        \draw (\yearcolumnwidth+0pt, -10pt) -- (\yearcolumnwidth+10pt, -10pt);
        \draw (\yearcolumnwidth+0pt, -\timelineheight+15pt) -- (\yearcolumnwidth+10pt, -\timelineheight+15pt);

        \pgfmathsetlengthmacro{\templength}{neg(add(multiply(subtract(\startyear, \startyear), divide(subtract(\timelineheight, 25), subtract(\tlendyear, \startyear))), 10))}
        \node[year] (year-\startyear) at (\yearcolumnwidth, \templength) {\startyear};
        \pgfmathsetlengthmacro{\templength}{neg(add(multiply(subtract(\tlendyear, \startyear), divide(subtract(\timelineheight, 25), subtract(\tlendyear, \startyear))), 10))}
        \node[year] (year-\tlendyear) at (\yearcolumnwidth, \templength) {\tlendyear};
    }

    \newcommand{\entry}[2]{
        \pgfmathtruncatemacro{\lastentrycount}{\entrycounter}
        \pgfmathtruncatemacro{\entrycounter}{\entrycounter + 1}

        \ifdim \lastentrycount pt > 0 pt%
            \node[entry] (entry-\entrycounter) [below of=entry-\lastentrycount] {##2};
        \else%
            \pgfmathsetlengthmacro{\templength}{neg(add(multiply(subtract(\startyear, \startyear), divide(subtract(\timelineheight, 25), subtract(\tlendyear, \startyear))), 10))}
            \node[entry] (entry-\entrycounter) at (\yearcolumnwidth+\rulecolumnwidth+10pt, \templength) {##2};
        \fi

        \ifnodeundefined{year-##1}{%
            \pgfmathsetlengthmacro{\templength}{neg(add(multiply(subtract(##1, \startyear), divide(subtract(\timelineheight, 25), subtract(\tlendyear, \startyear))), 10))}
            \draw (\yearcolumnwidth+2.5pt, \templength) -- (\yearcolumnwidth+7.5pt, \templength);
            \node[year] (year-##1) at (\yearcolumnwidth, \templength) {##1};
        }

        \draw ($(year-##1.east)+(2.5pt, 0pt)$) -- ($(year-##1.east)+(7.5pt, 0pt)$) -- ($(entry-\entrycounter.west)-(5pt,0)$) -- (entry-\entrycounter.west);
    }

    \newcommand{\plainentry}[2]{% plainentry won't print date in the timeline
        % #1 is the year
        % #2 is the entry text

        \pgfmathtruncatemacro{\lastentrycount}{\entrycounter}
        \pgfmathtruncatemacro{\entrycounter}{\entrycounter + 1}
        \ifdim \lastentrycount pt > 0 pt%
            \node[entry] (entry-\entrycounter) [below of=entry-\lastentrycount] {##2};
        \else%
            \pgfmathsetlengthmacro{\templength}{neg(add(multiply(subtract(\startyear, \startyear), divide(subtract(\timelineheight, 25), subtract(\tlendyear, \startyear))), 10))}
            \node[entry] (entry-\entrycounter) at (\yearcolumnwidth+\rulecolumnwidth+10pt, \templength) {##2};
        \fi

        \ifnodeundefined{invisible-year-##1}{%
            \pgfmathsetlengthmacro{\templength}{neg(add(multiply(subtract(##1, \startyear), divide(subtract(\timelineheight, 25), subtract(\tlendyear, \startyear))), 10))}
            \draw (\yearcolumnwidth+2.5pt, \templength) -- (\yearcolumnwidth+7.5pt, \templength);
            \node[year] (invisible-year-##1) at (\yearcolumnwidth, \templength) {};
        }

        \draw ($(invisible-year-##1.east)+(2.5pt, 0pt)$) -- ($(invisible-year-##1.east)+(7.5pt, 0pt)$) -- ($(entry-\entrycounter.west)-(5pt,0)$) -- (entry-\entrycounter.west);
    }

    \begin{tikzpicture}
        \tikzstyle{entry} = [%
            align=left,%
            text width=\entrycolumnwidth,%
            node distance=10mm,%
            anchor=west]
        \tikzstyle{year} = [anchor=east]
        \tikzstyle{timelinerule} = [%
            draw,%
            decoration={markings, mark=at position 1 with {\arrow[scale=1.5]{latex'}}},%
            postaction={decorate},%
            shorten >=0.4pt]

        \drawtimeline
}
{
    \end{tikzpicture}
    \let\startyear\@undefined
    \let\tlendyear\@undefined
    \let\yearcolumnwidth\@undefined
    \let\rulecolumnwidth\@undefined
    \let\entrycolumnwidth\@undefined
    \let\timelineheight\@undefined
    \let\entrycounter\@undefined
    \let\ifnodedefined\@undefined
    \let\ifnodeundefined\@undefined
    \let\drawtimeline\@undefined
    \let\entry\@undefined
}



%%%%%%%%%%%%%%%%%%%%%%%%%%%%%%%%%%%%%%%%
% Cores dos links
%%%%%%%%%%%%%%%%%%%%%%%%%%%%%%%%%%%%%%%%

% Veja o arquivos cores.tex se quiser ver que outras cores estão
% pré-definidas.  Utilizando o comando \hypersetup abaixo nós
% evitamos aquelas caixas vermelhas feias em volta dos links.

\input{cores}
\hypersetup{
  colorlinks=true,
  linkcolor=DarkScarletRed,
  citecolor=DarkScarletRed,
  filecolor=DarkScarletRed,
  urlcolor= DarkScarletRed
}



%%%%%%%%%%%%%%%%%%%%%%%%%%%%%%%%%%%%%%%%
% Informações sobre a monografia
%%%%%%%%%%%%%%%%%%%%%%%%%%%%%%%%%%%%%%%%

\title{Avaliação de Interação Humano-Computador: um estudo de caso para Bioinformática}

\orientador[a]{\prof \dr[a] Maria Emília Machado Telles Walter}{CIC/UnB}
%\coorientador[a]{\prof[a] \dr[a] Coorientadora}{MAT/UnB}
\coordenador{\prof \dr Rodrigo Bonifácio de Almeida}{CIC/UnB}
\diamesano{08}{Julho}{2016}

\membrobanca{\dr[a] Tainá Raiol Alencar}{Fiocruz}
\membrobanca{\prof \dr Rodrigo Bonifacio de Almeida}{CIC/UnB}

\autor{Gabriella de Oliveira}{Esteves}
\CDU{004.4}

\palavraschave{Bioinformática, Redes Metabólicas, Grafo, Método de Análise de Computabilidade, Interação Humano-Computador}
\keywords{Bioinformatics, Metabolic Networks, Graph, Communicability Evaluation Method, Human-Computer Interaction}



%%%%%%%%%%%%%%%%%%%%%%%%%%%%%%%%%%%%%%%%
% Texto
%%%%%%%%%%%%%%%%%%%%%%%%%%%%%%%%%%%%%%%%

\begin{document}
  \maketitle
  \pretextual

  %\begin{dedicatoria}
  
  %\end{dedicatoria}

  \begin{agradecimentos}
  Agradeço à professora Maria Emília e ao pesquisador Waldeyr Mendes, pela paciência e orientação. Agradeço à minha família pelo suporte e dedicação em me ajudar nessa fase final da graduação. Agradeço à minha família e amigos pelo apoio.
  \end{agradecimentos}

  \begin{resumo}
  O \textit{2Path} é um banco de dados de vias metabólicas de terpenoides. Ele possui um sistema que recebe um arquivo FASTA de múltiplas sequências de nucleotídeos de enzimas e verifica se elas estão armazenadas no banco. Caso alguma enzima esteja no banco, então ela está presente no organismo em questão, representado pelo arquivo FASTA. O objetivo deste trabalho é apresentar um projeto de interface para o banco \textit{2Path}, em que seja possível fazer \textit{upload} de arquivos FASTA com enzimas, além de permitir a pesquisa por enzimas e vias metabólicas. A interface proposta foi testada e aperfeiçoada de acordo com o Método de Avaliação de Comunicabilidade, proposto na área de Interação Humano-Computador. Três pessoas participaram da avaliação da interface, o que permitiu a observação de diversas falhas de comunicabilidade. A partir dessa avaliação, a interface foi aperfeiçoada de acordo com certos critérios de qualidade.
  \end{resumo}

  \selectlanguage{american}
  \begin{abstract}
  The 2Path is a terpenoid metabolic pathway database. It has a system that receives enzymes as an input, represented by a multi-FASTA file nucleotide sequences, and verifies if they are stored in the database. If so, then are in the organism, represented by the FASTA file. The goal of this project is to propose an interface to the 2Path database such that users can upload a FASTA file and search for enzymes and metabolic pathways in the database. The proposed interface will be tested and improved according to the Communicability Evaluation Method, proposed in the Human-Computer Interaction field. Three people participated in the evaluation of the interface, which allowed the observation of several failures of communicability. From this evaluation, the interface has been improved according to certain quality criteria.
  \end{abstract}
  \selectlanguage{brazil}

  \tableofcontents
  \listoffigures
  \listoftables

  \textual 
  \chapter{Introdução}

\section{História da Genética}

\indent O estudo do núcleo celular começou no século XIX, em um laboratório na Alemanha, com o objetivo de catalogar as substâncias químicas presentes nas células sanguíneas do ser humano. Como naquela época as pesquisas eram mais voltadas ao citoplasma - fluido pastoso que constitui a célula, o bioquímico suíço Friedrich Miescher foi o pioneiro no estudo do núcleo. Ele quem descobriu a substância nucleína composta por carbono, hidrogênio, oxigênio, nitrogênio e fósforo (ausênte nas proteínas), que mais tarde chamaram de ácido desoxirribonucléico, ou DNA. \\

\indent No início do século XX, o geneticista estadunidense Thomas Morgan liderou uma equipe de estudantes e realizou vários experimentos em \textit{Drosophila melanogaster} - espécie de mosca, com a finalidade de compreender a hereditariedade a partir de genes transmitidos aos organismos em desenvolvimento. Esta pesquisa foi fundamental para demonstrar exerimentalmente a Teoria Cromossômica da Hereditariedade (Sutton-Boveri, 1902), que assumem várias suposições como verdade, dentre elas: Os genes estão localizados em cromossomos; Os cromossomos formam pares de homólogos; Destes pares, um tem origem paterna, o outro tem origem materna. Tais hipóteses são baseadas nos experimentos caseiros do botânico Gregor Mendel, que após 8 anos de experimentos (1856-1863), publicou seu paper na Nature Research Society of Brünn. Nele, Mendel introduz conceitos como dominância, fator recessivo, hereditariedade, segregação dos fatores e transmissão independente dos genes. O trabalho de Morgan e sua equipe rendeu-lhe um Prêmio Nobel de Fisiologia ou Medicina em 1933. \\

\indent No início dos anos 50, uma química britânica chamada Rosalind Frankling usou a técnica de difração de raios-X para determinação da estrutura da biomolécula do DNA e concluiu que sua forma era helicoidal. Seu trabalho foi empregado nos experimentos de dois pesquisadores, Francis Crick e James Watson, em um laboratório em Cambridge, na Inglaterra. No mesmo ano, a dupla decifrou a estrutura do DNA: duas longas fitas enroladas uma na outra em espiral para a direita, ligadas por pares de bases complementares, formando o que chamaram de dupla-hélice. Apesar da grande descoberta, isto não era o suficiente para entender como eram produzidas as proteínas, portanto os cientistas mudaram o foco das pesquisas para o RNA, uma vez que sabiam o quanto sua concentração aumentava sempre que as células começavam a produzir proteínas. Em 1958, Crick e Watson anunciaram mais uma descoberta: A partir do DNA, o processo de \textit{transcrição} fornece uma fita de RNA, que por sua vez, a partir do processo de \textit{tradução}, fornecem a proteína. Esta sequência de processos ficou conhecida como Dogma Central da biologia molecular. \\

\textbf{CITAR: O polegar do violinista}

\vspace{1cm}
 \begin{figure}[h!]
     \centering
     \includegraphics[scale=0.4]{JamesWatsonAndFrancisCrick.jpg}
     \caption{James Watson e Francis Crick}
     \label{fig:JamesWatsonAndFrancisCrick}
 \end{figure}
\vspace{1cm}

\indent Bioinformática. Margaret Dayhoff e Walter Goad \\.\\.\\.\\.\\.\\.\\.\\

%  -------------------------------------------------------------
%  -------------------------------------------------------------

\indent A linha do tempo abaixo tem o objetivo de auxiliar na localização temporal da história da biologia molecular e da bioinformática ao passo que apresentam as datas de nascimento dos principais pesquisadores da área.

%  -------------------------------------------- 
%  -------------------------------------------- LINHA DO TEMPO
%  -------------------------------------------- 

\begin{timeline}{1809}{1933}{2cm}{2.5cm}{10cm}{1\textheight}
\entry{1809}{$\star$ Charles Darwin, Inglaterra}
\entry{1822}{$\star$ Gregor Mendel, Áustria}
\entry{1844}{$\star$ Friedrich Miescher, Suíça}
\entry{1859}{Darwin publica livro A Origem das Espécies} %On the Origin of Species by Means of Natural Selection,
% or the Preservation of Favoured Races in the Struggle for Life.
%  6º ed. It is in this edition that the word 'evolution' occurs for the first time.
% http://darwin-online.org.uk/content/frameset?pageseq=80&itemID=A1&viewtype=text
% Genes sobreviventes demoram a se espalhar pois são aleatóriamente selecionados
\entry{1865}{Mendel apresenta pela primeira vez seu trabalho com ervilhas}
% http://www.wired.com/2010/02/0208gregor-mendel-reads-paper/
\entry{1866}{$\star$ Thomas Morgan, Estados Unidos}
\entry{1871}{Miescher publicou seu paper sobre a nucleína}
%\entry{1882}{$\dagger$ Charles Darwin}
%\entry{1884}{$\dagger$ Gregor Mendel}
\entry{1890}{$\star$ Hermann Muller, Estados Unidos}
%\entry{1895}{$\dagger$ Friedrich Miescher} %%
\entry{1916}{$\star$ Francis Crick, Inglaterra}
\entry{1925}{$\star$ Margaret Dayhoff, Estados Unidos}
\entry{1925}{$\star$ Walter Goad, Estados Unidos}
\entry{1928}{$\star$ James Watson, Estados Unidos}
\entry{1933}{Thomas Morgan recebe Prêmio Nobel de Fisiologia ou Medicina por mostrar experimentalmente a Teoria Cromossômica da Hereditariedade}
%\entry{1945}{$\dagger$ Thomas Morgan}
%\entry{1967}{$\dagger$ Hermann Muller}
%\entry{2004}{$\dagger$ Francis Crick}
\end{timeline}


% ADICIONAR DADOS: http://www.ncbi.nlm.nih.gov/pmc/articles/PMC2898077/figure/F2/

%  -------------------------------------------- 
%  -------------------------------------------- LINHA DO TEMPO
%  -------------------------------------------- 


\section{Definição do Problema}

\indent 
Contruir uma visualização interativa de redes metabólicas aramazenadas em banco de dados de grafos que permita ao pesquisador explorar os aspectos biológicos do organismo estudado.



\section{Justificativa}

\indent 
% Falar da quantidade de dados que e muito grande para ser examinada e o quanto um sistema com busca e uma visualizacao interativa
% pode auxiliar o pesquisador.

Atualmente, a quantidade de dados <<<<>>>> estudados pelos pesquisadores é extensa e complexa. Uma maneira de amenizar o esforço feito para analisá-los e compreendê-los é oferecer uma ferramenta que aproxime o usuário (pesquisador) e os dados em forma de grafo(redes metabólicas). Esta ferramenta deverá permitir que o usuário visualize e interaja com os dados dinamicamente, além de disponibilizar mecanismos de busca em grafos, úteis para sua pesquisa.


\section{Objetivo}

\indent 
Constrir um sistema que acesse redes metabólicas armazenadas em bancos de dados em grafo e gere uma visualização interativa
\begin{itemize}
 \item Implementar uma busca das vias metabólicas de interesse a apartir de parâmetros informados pelo pesquisador no sistema
 \item Recuperar a informação desejada e exibí-la para o pesquisador de forma ergonômica
 \item Implementar algoritmos de busca em grafos para recuperar a informação solicitada e/ou sugerir informação relevante
\end{itemize}

\section{Descrição dos Capítulos}

\indent No Capítulo 1 fez-se uma breve introdução à história da biologia molecular e da bioinformática. No Capítulo 2 são estabelecidas as principais definições utilizadas neste trabalho mais profundamente, tais como ácidos nucléidos, biomoléculas gerais que originam o DNA e o RNA; a proteína, macromolécula extensa, formada por um processo complexo chamado síntese de proteína; código genético, listagem do arranjo de bases nitrogenadas que formam aminoácidos, que por sua vez compõem a proteína; Neste capítulo também são descritos os processos de sequenciamento de proteínas, na subseção de bioinformática e os desafios enfrentados nessa área. \\

\indent O Capítulo 3 apresenta uma estrutura chamada Redes metabólicas, estrutura de dados extremamente complexas que existem para auxiliar o pesquisador biólogo a entender reações intracelulares, bem como determinar propriedades fisiológicas e bioquímicas das células. A construção destas redes é possível pos existe sequenciamento do genoma do organismo estudado. O Capítulo 4 propõe um banco de dados não relacional (NoDB) em grafos como maneira de armazenar estas redes metabólicas. Nele é descrito todo o conceito de NoDB, e é apresentado aquele utilizado neste trabalho: banco de dados neo4j. \\

\indent No Capítulo 5 são exibidos os resultados da implementação do programa e no Capítulo 6, as conclusões tiradas a parir da análise dos dados. O Capítulo 7 expõe os problemas enfrentados, bem como sugestões de melhorias e trabalhos futuros. Por fim, o Capítulo 8 apresenta uma tabela do cronograma da execução deste trabalho.






 
  \chapter{Avaliação em Interação Humano-Computador}

\indent Atualmente a tecnologia está presente mais do que nunca em grande parte das atividades das pessoas, modernizando as casas, ruas, escolas, diferentes ambientes de trabalho, integrando-se na vida pessoal de cada um, se difundindo cada vez mais rápido. Essas tecnologias são cada uma um sistema computacional distinto que interage a sua maneira com os usuário. Elas podem interferir direta e/ou indiretamente nos processos de comunicação e informação dos indivíduos \cite{tics}. 

\indent Para satisfazer certos critérios de qualidade, estes sistemas devem possuir algumas características que facilitem seu uso para seus usuários específicos. Ainda, vários métodos de avaliação podem ser aplicados sobre eles, cada um com um foco diferente a ser analisado \cite{IHCbook}. Em Ciência da Computação, a área de Interação Humano-Computador (IHC) é responsável por verificar a qualidade destes sistema mensurando certas propriedades e avaliar o impacto dos mesmos na vida dos usuários. 

\indent Na Seção \ref{cbIHC} serão descritos os conceitos básicos de IHC bem como os critérios de qualidade avaliados nos sistemas computacionais. Na Seção \ref{avaObs} será apresentado um modelo de avaliação de sistemas computacionais, chamado Avaliação Através de Observação. Um método que utiliza este modelo, chamado Método de Avaliação de Comunicabilidade, será descrito detalhadamente também nessa seção e será utilizado neste projeto.
 
\section{Conceitos Básicos de IHC} \label{cbIHC}

\indent Esta seção apresenta os componentes básicos utilizados na matéria interdisciplinar Interação Humano-Computador, envolvidos na interação entre usuários e algum sistema computacional. A \textbf{interação} engloba todo o contato realizado pelo usuário com o objetivo de exercer uma tarefa através do sistema. Segundo John Kammersgaard, 1988, existem quatro perspectivas de interação usuário-sistema \cite{IHCbook}:

\begin{itemize}
\item[1] \textbf{Sistema}: Usuário conhece linguagem específica voltada para o sistema e seu objetivo é a transmissão correta dos dados da maneira mais rápida possível;
\item[2] \textbf{Parceiro de discurso}: Usuário interage com o sistema através de uma inteligência artificial que personifica um interlocutor especialista naquilo que usuário procura;
\item[3] \textbf{Mídia}: Usuários interagem entre si por meio do sistema, que pode oferecer diversos recursos como jogos e portal de notícias, porém é focado na qualidade da comunicação entre pessoas;
\item[4] \textbf{Ferramenta}: Usuário utiliza o sistema de maneira automática como instrumento de propósito geral para realizar suas tarefas. 
\end{itemize}

\indent Em um sistema interativo, a \textbf{interface} é responsável por manter o contato motor (como a \textit{webcam} e o teclado), perceptivo (como o monitor) e conceitual (interpretação do contato físico) do usuário \cite{IHCbook}. 

\indent O conjunto dos elementos da interface que expõem seu funcionamento de maneira implícita é chamado de \textbf{\textit{affordance}} \cite{IHCbook}. Esta palavra não possui tradução para o português, mas entende-se que significa "reconhecimento", pois uma vez que o usuário é levado a realizar os passos corretos para cumprir seus objetivos em uma interface, mesmo sem nunca ter interagido com ela antes, considera-se que ele já conhecia algumas características que o sistema oferece e obteve respostas já esperadas. Nesse sentido, a \textit{affordance} é geralmente uma característica muito desejável nos sistemas. Na Figura \ref{fig:Affordance} é possível perceber três graus diferentes de \textit{affordance} na \ref{fig:Affordance}.

\begin{figure}[!h]
    \centering
    \includegraphics[width=0.4\textwidth]{affordance.png}
    \caption{Exemplo de tipos de caixa de texto que representam o mesmo campo (Nome) porém com níveis de \textit{affordance} diferentes, sendo que o último apresenta melhor o objetivo da caixa e, portanto, ela é reconhecida mais facilmente. Adaptado de \cite{affordance}.}
    \label{fig:Affordance}
\end{figure} 

\indent Em IHC, a qualidade de um certo sistema está fortemente relacionada à sua interface e interação \cite{IHCbook}, pois para que os usuários aproveitem o sistema por completo, eles deve estar confortáveis com o ambiente. As características da interação e interface que quantificam a qualidade de um sistema são chamadas de critérios de qualidade, apresentados a seguir:

\begin{itemize}
\item[1] \textbf{Usabilidade}: Medida de complexidade no aprendizado do uso da interface para atingir objetivos de maneira eficiente (no menor tempo possível, utilizando o menor número de recursos do sistema) e eficaz (de modo que o usuário execute as tarefas o mais automático possível, sem apresentar dúvidas a respeito da interface);
\item[2] \textbf{Experiência do usuário}: Medida de satisfação do usuário em relação ao sistema;
\item[3] \textbf{Acessibilidade}: Medida da flexibilidade do sistema, ou seja, da capacidade de usuários interagirem com o mesmo;
\item[4] \textbf{Comunicabilidade}: Medida da capacidade de transmissão das intenções do projetista do sistema para o usuário por meio da interface. %Exemplo: através de analogias
\end{itemize}

\indent Ao avaliar a qualidade de uso de um sistema, é necessário escolher um tipo de avaliação apropriado para o objeto estudado e para o critério de qualidade em foco, se existir. A avaliação através de inspeção, por exemplo, tenta conhecer as possíveis respostas do usuário em certas situação no sistema ainda em produção. Essa avaliação possui três métodos distintos, cada um com um objetivo diferente. O método pode ser heurístico (focada em problemas de usabilidade durante um processo interativo), cognitivo (focada em avaliar a facilidade de aprendizado de um sistema interativo) ou inspeção semiótica (focado em avaliar a quantidade de informação transmitida do projetista do sistema para o usuário).

\indent Outro tipo de avaliação bastante utilizado é o de avaliação por observação, o qual será utilizado neste trabalho. Esse tipo de avaliação considera problemas reais que os usuários (ou participantes com o mesmo perfil dos usuários) enfrenta ao utilizar o sistema. Existem três métodos de avaliação baseados em avaliação por observação: teste de usabilidade (focado em avaliar a usabilidade através de registros de performance e opiniões dos usuários), método de avaliação de comunicabilidade (focado em avaliar a quantidade de informação que chega ao usuário através do sistema) e prototipação em papel (focado em avaliar a usabilidade através de simulações do sistema em papel).  

\section{Avaliação Através de Observação} \label{avaObs}

\indent Para manter os critérios básicos de qualidades, é importante que exista algum tipo de avaliação do sistema antes de sua entrega ao(s) usuário(s). Na perspectiva de quem desenvolve o sistema, a avaliação deve verificar se o sistema recebe as entradas, processa os dados e gera o resultado na saída corretamente. Por outro lado, na perspectiva do usuário, a avaliação deve verificar como o comportamento da interface afeta a experiência de uso do sistema \cite{IHCbook}, recaindo nos quatro critérios de qualidade descritos na Seção \ref{cbIHC}. A etapa de avaliação pode ocorrer durante (\textbf{avaliação formativa}) ou após (\textbf{avaliação somativa}) o processo de desenvolvimento do sistema. Na primeira opção, o foco é a identificação de possíveis problemas, com a vantagem do baixo custo de correção. Na segunda opção, o foco é a verificação dos níveis de qualidade do protótipo de escopo definido \cite{IHCbook}.

\indent Os tópicos mais avaliados em IHC são os problemas na interação e na interface do sistema, que são classificados de acordo com a frequência com que ocorrem, com sua gravidade ou com os quatro critérios de qualidade. Um exemplo de questionário associado a esse tópico de avaliação encontra-se na parte de consolidação dos resultados, como será descrito a seguir nessa seção.

\indent A fundamentação dos métodos de avaliação é dada pela teoria da Engenharia Semiótica, focada em dois tipos de comunicações: usuário-sistema e projetista-usuário através do sistema. Esse último tipo de comunicação recebe o nome de \textbf{metacomunicação}, uma vez que é feita indiretamente através da interface criada pelo projetista para o usuário. Segundo esta teoria, toda aplicação computacional é um artefato de metacomunicação por onde o projetista, como interlocutor, comunica-se com o usuário.

\indent A \textbf{metamensagem} é a mensagem transmitida via metacomunicação. Em IHC, a metamensagem possui um padrão fixo e generalizado, que pode ser utilizado para facilitar o entendimento dos problemas do usuário, o que pode ser feito para solucioná-los e como fazê-lo. O texto a seguir representa esta mensagem e os campos entre os símbolos ``<'' e ``>'' representam um espaço a ser ocupado de acordo com o sistema desenvolvido:

\begin{quote}
Este é o meu entendimento, como \textit{<projetista do sistema>}, de quem você, \textit{<usuário específico / grupo de usuários>}, é, do que aprendi que você quer ou precisa \textit{<fazer / executar / visualizar / se comunicar>}, de que maneiras \textit{<prefere / é mais natural>} fazer, e \textit{<por quê / por quê não outras>}. O \textit{<nome do sistema>}, portanto, é o sistema que projetei para você, e \textit{<esta navegação específica>} é a forma como você pode ou deve utilizá-lo para alcançar uma gama de objetivos que se encaixam nesta visão.
\end{quote}

\indent O projetista de um sistema cria signos para se comunicar com o usuário através da interface. Estes signos podem ser estáticos (como rótulos, imagens, botões, cores), dinâmicos (como ações, transições de tela, notificações) ou metaliguísticos (informações referentes à outros signos). A Engenharia Semiótica também estuda o processo de significação dos signos, denominado semiose. A semiose abrange tudo o que é percebido pelos seres humanos e produz sensação, na mente dos intérpretes. Esse é um processo que transforma os fenômenos perceptíveis através dos sentidos em experiências dos indivíduos \cite{semiose01}. 

\indent Em IHC, a semiose é utilizada para compreender as ações tomadas, bem como as reações de resposta, pelos usuários que interagem com sistemas cobertos de signos. Assim, é possível elaborar um perfil semiótico do sistema para retratar os problemas de comunicabilidade, e, para isso, a metamensagem pode ser utilizada como base.

\indent Existem dois métodos de avaliação baseados em Engenharia Semiótica. Quando o objetivo é avaliar a quantidade de emissão de metacomunicação, utiliza-se o Método de Inspeção Semiótica \cite{IHCbook}. Quando é avaliar a quantidade de recepção de metacomunicação, utiliza-se o Método de Avaliação de Comunicabilidade \cite{IHCbook}. O primeiro método, porém, tem como objetivo antever possíveis problemas causados por decisões do projetista, logo não envolve obstáculos reais dos usuários. Já o segundo método visa compreender as dificuldades enfrentadas pelos usuários enquanto navegam de fato pelo sistema. Nesse sentido, o Método de Avaliação de Comunicabilidade será utilizado neste trabalho.


\subsection{Método de Avaliação de Comunicabilidade} \label{MAC}

\indent O Método de Avaliação de Comunicabilidade (MAC) é um método qualitativo de avaliação somativa, cujo objeto são os problemas na interação e interface do sistema, com foco na percepção do usuário. Ele consiste em uma análise de gravações em vídeo de pessoas utilizando o sistema com base em 13 etiquetas (expressões linguísticas que caracterizam ruptura na comunicação) e é composto por cinco etapas\cite{IHCbook}: preparação, coleta de dados, interpretação, consolidação dos resultados e relato dos resultados. As pessoas filmadas não são necessariamente os usuários do sistema, elas podem ser representantes de usuários que possuem o mesmo perfil daqueles para qual o sistema é direcionado. 

\subsubsection{Preparação}

\indent Nesta etapa, o avaliador é responsável por definir o perfil dos usuários, selecionar os representantes dos mesmos e elaborar o ambiente e tarefa realizada por eles. O material da gravação é preparado e verificado fazendo-se um teste piloto. Um termo de consentimento da avaliação deve ser redigido para assinarem o avaliador e cada representante. Devem ser elaborados dois questionários: um pré-teste, para coletar informações de cada representante tais como conhecimento sobre o domínio do sistema e um pós-teste, para coletar informações referentes à opinião dos participantes sobre suas experiências com o sistema.

\subsubsection{Coleta de Dados}

\indent Aqui o avaliador deve receber os representantes de usuários, explicar o procedimento da tarefa realizada, entregar uma via do termo de consentimento para cada representante assinar, já com a assinatura do avaliador. Os questionários pré-teste são entregues e cabe ao avaliador decidir se é necessário ler em voz alta as perguntas, caso alguma possa conter mais de uma interpretação. Não é especificado tempo mínimo nem máximo para esta tarefa. Ao final do questionário, o avaliador posiciona os equipamentos de gravação perto de cada participante, de maneira que a imagem gravada seja a tela, o \textit{mouse}/\textit{mousepad}, o teclado e as mãos dos mesmos, e o som captado seja apenas a voz dos mesmos. 

\indent Uma técnica muito conhecida de coleta de dados na gravação é chamada \textit{think aloud}, onde os participantes devem relatar em voz alta tudo aquilo que estão pensando em relação ao sistema, como execução, planejamento de execução, reação às respostas da interface, etc. O resultado final facilita bastante a análise do avaliador, porém o ato de falar enquanto desempenha a tarefa pode descentralizar o pensamento dos participantes. Uma vez que estes sintam-se seguros em realizar as tarefas ao mesmo tempo em que se expressam verbalmente, os avaliadores terão muito mais dados (etiquetas) para interpretar na fase seguinte.

\indent Terminada a gravação, ocorre uma entrevista para coletar a opinião de todos os participantes em relação ao sistema, bem como tirar dúvidas sobre seus desempenhos.

\subsubsection{Interpretação}

\indent Nesta etapa o avaliador assiste todos os vídeos contabilizando certas expressões linguísticas chamadas de etiquetas apresentadas na Tabela \ref{tabEtiquetas}. Deve-se levar em consideração os perfis dos usuários traçados pelo questionário pré-teste, bem como as entrevistas pós-teste, uma vez que as experiências passadas de cada um pode afetar seu comportamento no sistema. 

\indent As expressões linguísticas possuem na maioria conotação ruim com respeito ao sistema, tais como indagação sobre seu comportamento e aversão às respostas. Observando os vídeos, o avaliador pode perceber as barreiras de comunicabilidade e gargalos do sistema do ponto de vista do usuário, chamados de \textbf{falha de comunicabilidade}. Quando o número de participantes é muito baixo, o avaliador pode fazer a interpretação ao mesmo tempo em que coleta dados, sem precisar de gravação.

\begin{table}[]
\centering
\caption{Descrição das etiquetas, de acordo com a falha de comunicação que representam \cite{IHCbook}.}
\label{tabEtiquetas}
\begin{tabular}{|l|l|c|}
\hline
\multicolumn{3}{|l|}{\cellcolor[HTML]{DFDFDF}\specialcell{\textbf{Falhas de comunicação completas: efeito obtido é inconsistente com}\\\textbf{a intenção comunicativa do usuário}}} \\\hline
\textbf{aspecto semiótico}  & \textbf{característica específica} & \textbf{etiqueta} \\ \hline
\multirow{-2}{*}{\specialcell{O usuário termina uma\\semiose malsucedida,\\mas não inicia outra\\para obter o resultado\\esperado}}
    &  \specialcell{porque, mesmo percebendo que não obteve o\\resultado esperado, não possui mais recursos,\\capacidade ou vontade de continuar tentando}&Desisto \\ \cline{2-3} 
	&  \specialcell{porque não percebe que não obteve o\\resultado esperado}&\specialcell{Para mim\\está bom...} \\ \hline
		
		
\multicolumn{3}{|l|}{\cellcolor[HTML]{DFDFDF}\specialcell{\textbf{Falhas de comunicação parciais: o efeito obtido é somente parte do}\\\textbf{efeito pretendido de acordo com a intenção do usuário}}} \\ \hline
\textbf{aspecto semiótico}  & \textbf{característica específica} & \textbf{etiqueta} \\ \hline
\multirow{-2}{*}{\specialcell{O usuário abandona\\uma semiose antes de\\obter o resultado espe-\\rado, e inicia outra\\com o mesmo propósito}}
    & \specialcell{porque, embora entenda a solução de IHC\\proposta, prefere seguir por outro caminho no\\momento} & \specialcell{Não,\\obrigado} \\ \cline{2-3} 
    &  porque não entende a solução de IHC proposta & \specialcell{Vai de\\outro\\jeito}  \\ \hline

    
\multicolumn{3}{|l|}{\cellcolor[HTML]{DFDFDF}\specialcell{\textbf{Falhas de comunicação temporárias: o efeito parcial do processo de}\\\textbf{interpretação (semiose) e de comunicação (interação) do usuário é}\\\textbf{inconsistente e incoerente com sua intenção de comunicação}}}   \\ \hline
\textbf{aspecto semiótico}  & \textbf{característica específica} & \textbf{etiqueta} \\ \hline

\multirow{3}{*}{\specialcell{O usuário interrompe\\temporariamente sua\\semiose}}
   &\specialcell{porque não encontra uma expressão\\apropriada para sua intenção de comunicação}& Cadê?\\ \cline{2-3} 
   &\specialcell{porque não percebe ou não entende a\\expressão do sistema (preposto do projetista)}& \specialcell{Ué, o que\\houve?} \\ \cline{2-3} 
   &\specialcell{porque não consegue formular sua próxima\\intenção de comunicação}& E agora? \\ \hline
   
\multirow{3}{*}{\specialcell{O usuário percebe que\\seu ato comunicativo\\não foi bem-sucedido}}
   &\specialcell{porque percebeu que havia "falado" algo no\\contexto errado}&\specialcell{Onde\\estou?} \\ \cline{2-3} 
   & porque percebeu que havia "falado" algo errado & Epa!\\ \cline{2-3} 
   &\specialcell{porque não obteve o resultado esperado depois\\de conversar com o sistema (preposto do\\projetista) por algum tempo, alternando vários\\turnos de fala com ele}&\specialcell{Assim não\\dá.}\\ \hline
   
\multirow{4}{*}{\specialcell{O usuário procura com-\\preender o ato comuni-\\cativo do sistema\\(preposto do projetista)}}
   &através da metacomunicação implícita & \specialcell{O que é\\isto?}\\ \cline{2-3} 
   &através da metacomunicação explícita & Socorro!\\ \cline{2-3} 
   &\specialcell{testando várias hipóteses sobre o significado do\\que o sistema comunicou}&\specialcell{Por que\\não\\funciona?} \\ \hline   
\end{tabular}
\end{table}



\indent Após a análise dos dados, o avaliador deve redefinir um protótipo da próxima versão do sistema já com a solução para os problemas mais simples e listar todos os obstáculos na interface que podem ser corrigidos.


\subsubsection{Consolidação dos Resultados}

\indent Este é o momento em que o avaliador busca diferenciar as características do grupo das características individuais observando a recorrência das etiquetas em certos pontos da navegação no sistema. Pode-se atribuir significado às etiquetas de acordo com frequência de uso em um certo contexto, com sequência de uso e com o nível de problemas, por exemplo. Além disso, como o foco desta avaliação é verificar os problemas relacionados à interação e interface, nesta epata devem ser respondidas as seguintes questões:

\begin{itemize}
\item O usuário consegue operar o sistema?
\item Ele atinge seu objetivo? Com quanta eficiência? Em quanto tempo? Após cometer quantos erros?
\item Que parte da interface e da interação o deixa insatisfeito?
\item Que parte da interface o desmotiva a explorar novas funcionalidades?
\item Ele entende o que significa e para que serve cada elemento da interface?
\item Ele vai entender o que deve fazer em seguida?
\item Que problemas de IHC dificultam ou impedem o usuário de alcançar seus objetivos?
\item Onde esses problemas se manifestam? Com que frequência tendem a ocorrer? Qual é a gravidade desses problemas?
\item Quais barreiras o usuário encontra para atingir seus objetivos?
\item Ele tem acesso a todas as informações oferecidas pelo sistema?
\end{itemize} 

\indent Respondidas as perguntas, o avaliador pode criar então um perfil semiótico para o sistema com base na metamensagem a seguir:

\begin{quote}
Este é o meu entendimento, como projetista, de \textbf{quem você, usuário, é} (1), do que aprendi que você \textbf{quer ou precisa fazer} (2), de \textbf{que maneiras prefere fazer} (3), e \textbf{por quê} (4). Este, portanto, é o sistema que projetei para você, e esta é \textbf{a forma como você pode ou deve utilizá-lo} (5) para alcançar uma gama de objetivos que se encaixam nesta visão.
\end{quote}

\indent Cada item numerado na meta mensagem deve fornecer as seguintes informações buscadas:

\begin{itemize}
\item[(1)] Qual é o perfil dos usuários?
\item[(2)] Quais são seus desejos e o que a metacomunicação realiza para satisfazê-los?
\item[(3)] Quais as maneiras de realizar seus desejos e de que maneira eles preferem fazer?
\item[(4)] O que os leva a ter esta preferência?
\item[(5)] Quão bem o conteúdo da metacomunicação é transmitidos aos usuários?
\end{itemize}


\subsubsection{Relato dos Resultados}

\indent Ao final, o avaliador deverá possuir material suficiente para apresentar aos \textit{stakeholders} responsáveis pelo desenvolvimento do sistema os seguintes resultados:

\begin{itemize}
\item Os objetos da avaliação, tais como interface do sistema e perfil dos usuários;
\item Breve descrição do Método de Avaliação de Comunicabilidade;
\item Quantidade e perfil dos avaliadores, bem como dos usuários ou dos participantes que os representam;
\item Descrição detalhada de todas as tarefas executadas pelos participantes;
\item Resultado das etiquetagens, em geral contabilizando as etiquetas por usuário e tarefa;
\item Lista de problemas de comunicabilidade encontrados;
\item Sugestões de melhoria para cada falha de comunicação;
\item Perfil semiótico do sistema, de acordo com a metamensagem padrão.
\end{itemize}


%http://acervodigital.ufpr.br/bitstream/handle/1884/36666/Luiz%20Augusto%20Sakakibara.pdf?sequence=1







% ==========================================================================================
% ==========================================================================================
% ==========================================================================================

\section{Projeto de Interface Centrado na Comunicação}

\indent Ao se montar um sistema com foco na comunicabilidade, o projetista deve elaborar um modelo conceitual de elementos do sistema computacional em desenvolvido. Esse modelo deve descrever detalhadamente como será realizada a interação usuário-sistema, os objetivos dos usuários e todas as possíveis tarefas que um usuário pode realizar. Além disso, para cada possível erro cometido pelo usuário no sistema, o projetista deve definir os métodos de recuperação e/ou prevenção dessas rupturas comunicativas \cite{IHCbook}.

\indent O objetivo desta seção é apresentar os mapas e tabelas utilizadas neste trabalho para a modelagem do sistema 2Path. 

\subsection{Tabela de Interações}

\indent Na Engenharia Semiótica, cabe ao projetista elaborar a metacomunicação do sistema, minimizando ao máximo qualquer tipo de falha de comunicação que pode levar a possível rupturas comunicativas. Para simplificar essa tarefa, pode-se representar todas as interações entre usuário e sistema em forma de conversação \cite{IHCbook}. Nesse sentido, o projetista pode criar uma tabela nos padrões da Tabela \ref{tabelaDeInteracao}, onde o usuário parece estar se comunicando com o projetista através de suas ações e as respostas obtidas.

\indent Cada conversa possui um tópico geral, como "Realizar compra" ou "Cadastrar Membro". Essas ações não serão executadas de fato no sistema, mas possuem o propósito de engajar o projetista a iniciar uma conversa mais específica. Os subtópicos que se estendem a partir de um tópico geral são chamados diálogos. Esses sim representam ações reais que podem precisar de alguma entrada do usuário (através de um formulário, por exemplo) e produzem respostas para dadas ações (mensagens de sucesso ou fracasso, por exemplo).

\indent 
\begin{table}
\centering
\caption{Modelo geral de representação da interação entre usuário (U) e projetista (P). Os signos representam o foco de cada conversa.} \label{tabelaDeInteracao}
\begin{tabular}{|l|l|}
\hline
{\cellcolor[HTML]{DFDFDF}\textbf{\specialcell{Tópico\\>Subtópico (diálogo)}}} &  {\cellcolor[HTML]{DFDFDF}\textbf{\specialcell{Falas e Signos\\U: Usuário e P: Projetista}}} \\ \hline
Ação com objetivo generalizado & \specialcell{U: Fala de usuário que precisa realizar\\um objetivo generalizado.} \\ \hline
> Ação com objetivo específico & \specialcell{U: Detalhes da requisição\\P: Resposta do sistema para o usuário.} \\ \hline
\end{tabular}
\end{table}

\subsection{Mapa de Objetivos}

\indent Um dos primeiros passos para se construir um projeto de interação é a definição dos objetivos dos usuários. Os objetivos podem ser finais ou instrumentais \cite{IHCbook}. Os objetivos finais são o motivo pelo qual o usuário está utilizando o sistema. Em um site de venda de roupas, por exemplo, o objetivo final pode ser "Comprar camiseta". Os objetivos instrumentais não são naturalmente percebidos pelos usuários. Eles servem para simplificar o objetivo final, realizando-o passo a passo. No mesmo exemplo, um objetivo instrumental pode ser "Adicionar camiseta no carrinho de compras".

\indent Existem dois tipos de objetivos instrumentais: direto e indireto \cite{IHCbook}. Os objetivos instrumentais diretos facilitam a realização do objetivo final de maneira imediata. No exemplo acima, a ação "Adicionar camiseta no carrinho de compras" é um objetivo instrumental direto. Já o indireto prepara atividades planejadas para o futuro. A ação "Adicionar cartão de crédito" pode ser considerada um objetivo instrumental indireto, pois não envolve diretamente a aquisição da camiseta, porém será uma informação necessária para a finalização da compra

\indent A escolha dos objetivos já envolve tomar decisões de interação usuário-sistema \cite{IHCbook}. Nesse sentido, ao criar o do mapa de objetivos representado de maneira generalizada na Figura \ref{fig:mapaDeObjetivosGeral}, o projetista já consegue imaginar qual deverá ser o foco da implementação do sistema para melhor atender as necessidades dos usuários. Esses usuários podem não possuir o mesmo perfil, e consequentemente, apresentar papéis diferentes. Assim, também cabe ao projetista definir as permissões de acesso de certos elementos em uma mesma páginas, de acordo com o perfil dos usuários e seus objetivos.

\begin{figure}[!h]
    \centering
    \includegraphics[width=1\textwidth]{mapaDeObjetivosGeral.png}
    \caption{}
    \label{fig:mapaDeObjetivosGeral}
\end{figure} 

\subsection{Tratamento de Rupturas na Comunicação}


\begin{itemize}
\item Prevenção passiva
\item Prevenção ativa
\item Prevenção apoiada
\item Recuperação apoiada
\item Captura de erro
\end{itemize}


\subsection{Modelagem de Tarefas}

\begin{figure}[!h]
    \centering
    \includegraphics[width=1\textwidth]{modelagemDeTarefas.png}
    \caption{}
    \label{fig:modelagemDeTarefas}
\end{figure} 

% =========================================================================================


  \chapter{Bioinformática}

\indent Conceitualização do algortimo.

Artigos:
Introdução
[introduction to bioinformatics for computer scientists]
  \chapter{Implementação}

\indent Implementação do algortimo.
  \chapter{Resultados}

Neste capítulo serão apresentados os primeiros resultados experimentais obtidos.

  \chapter{Conclusão e Trabalhos Futuros}

\section*{Conclusão}

\indent O 2path é uma base de conhecimento que preserva as principais características de biossíntese dos terpenos. Ele é capaz de processar um arquivo FASTA e retornar ao usuário um banco modificado, com informações sobre as sequências que geram certas enzimas do banco de dados. Assim, ao pesquisar por uma enzima ou via metabólica em um organismo, o usuário poderia obter todos esses dados. Muitas dessas informações, porém, ou não interessam aos biólogos, ou não devem aparecer em uma visualização interativa com a mesma importância que o que foi buscado de fato.

\indent Nesse sentido, a partir dos testes feitos pelo Método de Avaliação de Comunicabilidade de Interação Humano-Computador, observou-se que a simplicidade do objeto com que o usuário irá interagir é muito mais apreciado do que a quantidade de informação que ele pode obter. Uma interface com poucas páginas e textos é mais simples e, portanto, mais agradável. Uma página com mais informações é mais preferível que uma interface que demanda várias navegações e atualizações de telas.

\section*{Trabalhos Futuros}

\indent A partir da análise dos testes, percebeu-se um mudanças que poderiam ocorrer no banco de dados 2Path para aumentar a usabilidade de proposta de interface desenvolvida nesse projeto. Essa mudança seria no nó que representam as reações, e a sugestão é de incluir uma propriedade que indica a(s) enzima(s) que às cataliza(m). Com essa informação é possível representar facilmente no grafo as reações com as label do número EC das enzimas que às catalizam. Essa representação do grafo foi solicitado pela maioria dos usuário, que preferem imaginar uma reação como uma aresta do que como um nó, pois quem realiza as reações de fato são as enzimas.
  

  \postextual
  \bibliographystyle{plain}
  \bibliography{monografia}
  \appendix

\chapter{Questionários de dados pessoais dos biólogos}\label{questionario_dados_pessoais}

\includepdf[pages={1},scale=1]{capitulos/questionario_dados_pessoais.pdf}

\chapter{Tarefas realizadas pelos biólogos no sistema 2Path}\label{tarefas}

\includepdf[pages={1},scale=1]{capitulos/tarefas_2Path.pdf}

\chapter{Questionários de interface conforme o Método de Avaliação de Comunicabilidade}\label{questionario_interface}

\includepdf[pages={1},scale=1]{capitulos/questionario_interface.pdf}


\end{document}
\grid
\grid
