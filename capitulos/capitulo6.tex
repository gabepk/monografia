\chapter{Conclusão e Trabalhos Futuros}

\indent O \textit{2Path} é um banco de dados que armazena vias metabólicas com foco na biossíntese dos terpenos. Ele recebe como entrada organismos representados por arquivos FASTA de enzimas e retorna ao usuário um banco de dados para cada organismo. Assim, os usuários tem a opção de pesquisar por enzimas e vias metabólicas no banco de dados completo do \textit{2Path}, ou em cada banco de dados privado de um determinado organismo. As informações são passadas para o usuário na forma de um grafo. Existem muitas maneiras de adicionar informações em uma interface de visualização de dados, tais como legendas, títulos, rótulos, etc. 

\indent A partir dos testes feitos pelo Método de Avaliação de Comunicabilidade de Interação Humano-Computador com três biólogos, foi observado que muitas dessas informações, porém, ou não interessam aos biólogos, ou podem poluir a interface. A simplicidade do objeto com que o usuário irá interagir foi muito mais apreciado do que a quantidade de informação que ele pode obter. Uma interface com poucas páginas e textos foi considerada mais agradável por ser mais simples. Os usuários preferiram uma página com mais informações do que uma interface que demanda várias navegações e atualizações de telas. 

\indent Nesse sentido, uma nova interface mais compacta e mais legível foi proposta de acordo com os quatro de  qualidades: usabilidade, experiência de usuário, acessibilidade e comunicabilidade. A comunicabilidade foi analisada mais profundamente, pois é o foco do método utilizado na avaliação.

\indent A partir da análise do Método de Avaliação de Comunicabilidade, percebeu-se, também, mudanças que poderiam ocorrer no banco de dados \textit{2Path} para aumentar a usabilidade de proposta de interface desenvolvida nesse projeto. Essa mudança seria realizada nos nós que representam as reações, e a sugestão é de incluir uma propriedade que indica a(s) enzima(s) que as catalisa(m). Com essa informação, é possível representar facilmente no grafo as reações com os rótulos do número EC das enzimas que as catalisam. Essa representação do grafo foi solicitado pela maioria dos usuários, que preferem imaginar uma reação como uma aresta do que como um nó, pois quem realiza as reações de fato são as enzimas.