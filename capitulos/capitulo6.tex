\chapter{Conclusão e Trabalhos Futuros}

\section*{Conclusão}

\indent O 2path é uma base de conhecimento que preserva as principais características de biossíntese dos terpenos. Ele é capaz de processar um arquivo FASTA e retornar ao usuário um banco modificado, com informações sobre as sequências que geram certas enzimas do banco de dados. Assim, ao pesquisar por uma enzima ou via metabólica em um organismo, o usuário poderia obter todos esses dados. Muitas dessas informações, porém, ou não interessam aos biólogos, ou não devem aparecer em uma visualização interativa com a mesma importância que o que foi buscado de fato.

\indent Nesse sentido, a partir dos testes feitos pelo Método de Avaliação de Comunicabilidade de Interação Humano-Computador, observou-se que a simplicidade do objeto com que o usuário irá interagir é muito mais apreciado do que a quantidade de informação que ele pode obter. Uma interface com poucas páginas e textos é mais simples e, portanto, mais agradável. Uma página com mais informações é mais preferível que uma interface que demanda várias navegações e atualizações de telas.

\section*{Trabalhos Futuros}

\indent A partir da análise dos testes, percebeu-se um mudanças que poderiam ocorrer no banco de dados 2Path para aumentar a usabilidade de proposta de interface desenvolvida nesse projeto. Essa mudança seria no nó que representam as reações, e a sugestão é de incluir uma propriedade que indica a(s) enzima(s) que às cataliza(m). Com essa informação é possível representar facilmente no grafo as reações com as label do número EC das enzimas que às catalizam. Essa representação do grafo foi solicitado pela maioria dos usuário, que preferem imaginar uma reação como uma aresta do que como um nó, pois quem realiza as reações de fato são as enzimas.