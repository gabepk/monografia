\chapter{Introdução}

% O que é metabolismo

% Quais moléculas fazem parte do metabolismo

% Como o metabolismo tem sido representado computacionalmente (redes metabólicas)

% Como as redes metabólicas tem sido visualizadas? (estado da arte)

\section{Justificativa}
\indent Atualmente, a quantidade de dados ômicos estudados pelos pesquisadores é extensa e complexa. Uma maneira de amenizar o esforço feito para analisá-los e compreendê-los é oferecer uma ferramenta que aproxime o usuário (pesquisador) e os dados e a maneira mais natural é representar tais dados em forma de grafo (redes metabólicas). Esta ferramenta deverá permitir que o usuário visualize e interaja com os dados dinamicamente, além de disponibilizar mecanismos de busca em grafos, úteis para sua pesquisa.

\section{Problema}
\indent Construir uma visualização interativa de redes metabólicas armazenadas em banco de dados de grafos que permita ao pesquisador explorar os aspectos biológicos do organismo estudado.

\section{Objetivo}
\indent Construir um sistema que acesse redes metabólicas armazenadas em bancos de dados em grafo e gere uma visualização interativa
\begin{itemize}
 \item Implementar uma busca das vias metabólicas de interesse a a partir de parâmetros informados pelo pesquisador no sistema
 \item Recuperar a informação desejada e exibi-la para o pesquisador de forma ergonômica
 \item Implementar algoritmos de busca em grafos para recuperar a informação solicitada e/ou sugerir informação relevante
\end{itemize}

\section{Descrição dos Capítulos}
\indent No Capítulo 1 serão descritos os conceitos básicos de biologia molecular, metabolismo primário e secundário e os bancos de dados mais utilizados para armazenar as informações referentes às redes metabólicas. No Capítulo 2 serão apresentados detalhadamente quatro ferramentas de visualização de redes metabólicas: \textit{KEGG Pathway}, \textit{MetaCyc}, \textit{Reactome Browser} e \textit{Cytoscape}. \\ 
\indent Os Capítulos 3 e 4 já são relacionados ao sistema 2Path desenvolvido neste projeto. Enquanto o Capítulo 3 aborda o tema de interação humano-computador para a concepção de uma interface auto-explicativa e consistente, o Capítulo 4 descreve as linguagens e ambientes utilizados para a construção de tal. O Capítulo 5 especifica como são feitas as buscas por vias metabólicas no sistema pelo usuário e apresenta os resultados obtidos. \\
\indent O Capítulo 6 finaliza o trabalho com a conclusão e sugestão de trabalhos futuros.
