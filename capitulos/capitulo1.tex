\chapter{Avaliação em Interação Humano-Computador}

q isso? Cap 1

\section{Conceitos Básicos de IHC} \label{cbIHC}

\indent Esta seção apresenta os componentes básicos envolvidos na interação usuário-sistema 

\indent CONTINUAR

\indent A \textbf{interação} engloba todo o contato realizado pelo usuário com o objetivo de exercer uma tarefa através do sistema. Segundo John Kammersgaard, 1988, existem quatro perspectivas de interação usuário-sistema:
\begin{itemize}
\item[1] \textbf{Sistema}: Usuário conhece linguagem específica voltada para o sistema e seu objetivo é a transmissão correta dos dados da maneira mais rápida possível;
\item[2] \textbf{Parceiro de discurso}: Usuário interage com sistema através de uma inteligência artificial que personifica um interlocutor especialista naquilo que usuário procura;
\item[3] \textbf{Mídia}: Usuários interagem entre si por meio do sistema, que pode oferecer diversos recursos como jogos e portal de notícias, porém é focado na qualidade da comunicação entre pessoas;
\item[4] \textbf{Ferramenta}: Usuário utiliza sistema de maneira automática como instrumento de propósito geral para realizar suas tarefas. 
\end{itemize}

\indent Em um sistema interativo, a \textbf{interface} é responsável por manter o contato motor (como a webcam e o teclado), perceptivo (como o monitor) e conceitual (interpretação do contato físico) do usuário. O conjunto de características dos elementos da interface que exibem seu funcionamento é chamado de \textbf{\textit{affordance}}. 

\indent Em Interaão Humano-Computador, a qualidade de um certo sistema está fortemente relacionada à sua interface e interação. Os critérios de qualidade são:
\begin{itemize}
\item[1] \textbf{Usabilidade e experiência do usuário}: Medida de complexidade no aprendizado do uso da interface para atingir objetivos de maneira eficiente (no menor tempo possível, utilizando o menor número de recursos do sistema), eficaz (o mais automático possível) e com satisfação;
\item[2] \textbf{Acessibilidade}: Medida da flexibilidade do sistema, ou seja, da capacidade de usuários interagirem com o mesmo;
\item[3] \textbf{Comunicabilidade}: Medida da capacidade de transmissão das intenções do designer do sistema para o usuário por meio da interface. %Exemplo: através de analogias
\end{itemize}

\indent FINALIZAR

\section{Avaliação Através de Observação}

\indent Para manter os critérios básicos de qualidades\ é importante que exista algum tipo de avaliação do sistema antes de sua entrega ao(s) usuário(s). Na perspectiva de quem desenvolve o sistema, a avaliação deve verificar se o sistema recebe as entradas, processa os dados e gera o resultado na saída corretamente, enquanto que na perspectiva do usuário, a avaliação deve verificar como o comportamento da interface afeta a experiência de uso do sistema, recaindo nos quatro critérios de qualidade descritos na seção \ref{cdIHC}. [ ::: pag 286 ::: ] A etapa de avaliação pode se dar durante (avaliação formativa) ou após (avaliação somativa) o processo de design. Na primeira opção, o foco é a identificação de possíveis problemas com a vantagem do baixo custo de correção. Na segunda opção, o foco é a verificação dos níveis de qualidade do protótipo de escopo definido. [ ::: pag 294 ::: ]

\indent Os tópicos mais avaliados em IHC são os problemas na interação e na interface do sistema, que são classificados de acordo com a frequência com que ocorrem, com sua gravidade ou com os quatro critérios de qualidade. O questionário associado à este tópico de avaliação deve conter, por exemplo, as seguintes questões: [ ::: pág 293 CÓPIA FIEL DAS PERGUNTAS ::: ]
\begin{itemize}
\item O usuário consegue operar o sistema?
\item Ele atinge seu objetivo? Com quanta eficiência? Em quanto tempo? Após cometer quantos erros?
\item Que parte da interface e da interação o deixa insatisfeito?
\item Que parte da interface o desmotiva a explorar novas funcionalidades?
\item Ele entende o que significa e para que serve cada elemento da interface?
\item Ele vai entender o que deve fazer em seguida?
\item Que problemas de IHC dificultam ou impedem o usuário de alcançar seus objetivos?
\item Onde esses problemas se manifestam? Com que frequência tendem a ocorrer? Qual é a gravidade desses problemas?
\item Quais barreiras o usuário encontra para atingir seus objetivos?
\item Ele tem acesso a todas as informações oferecidas pelo sistema?
\end{itemize}

\indent A fundamentação por trás dos métodos de avaliação é dada pela teoria da engenharia semiótica, focada em dois tipos de comunicações: usuário-sistema e designer-usuário através do sistema (metacomunicação). Segundo esta teoria, toda aplicação computacional é um artefato de metacomunicação por onde o designer, como interlocutor, se comunica com o usuário. Quando o objetivo é avaliar a quantidade de emissão de metacomunicação, utiliza-se o Método de Inspeção Semiótica [::: pag 330 :::]. Quando é avaliar a quantidade de recepção de metacomunicação, utiliza-se o Método de Avaliação de Comunicabilidade. 

\subsection{Método de Avaliação de Comunicabilidade}


* quando: depois - avaliação somativa
* objeto: problemas na interação e interface
* foco: usuário --portanto--> recepção de metacomunicação ----> MAC
* tipos de dados: etiquetas --portanto--> quantitativos

* MAC: Metodo de avaliacao de comunicabilidade : engenharia semiótica : Raquel Prates (Bibiografia)
* think aloud
* videos com mão do usuario
* 13 etiquetas : barreiras de comunicabilidade : gargalos

Etiquetas: Mesma tabela do livro: interação humano computação: simone barbosa : portugue : campus

\begin{table}[]
\centering
\caption{My caption CÓPIA FIEL DA TABELA}
\label{my-label}
\begin{tabular}{|l|l|c|}
\hline
\multicolumn{3}{||l||}{\cellcolor[HTML]{9B9B9B}\specialcell{\textbf{Falhas de comunicação completas: efeito obtido é inconsistente com}\\\textbf{a intenção comunicativa do usuário}}} \\\hline
\textbf{aspecto semiótico}  & \textbf{característica específica} & \textbf{etiqueta} \\ \hline
\multirow{-2}{*}{\specialcell{O usuário termina uma\\semiose malsucedida,\\mas não inicia outra\\para obter o resultado\\esperado}}
    &  \specialcell{porque, mesmo percebendo que não obteve o\\resultado esperado, não possui mais recursos,\\capacidade ou vontade de continuar tentando}&Desisto \\ \cline{2-3} 
	&  \specialcell{porque não percebe que não obteve o\\resultado esperado}&\specialcell{Para mim\\está bom...} \\ \hline
		
		
\multicolumn{3}{||l||}{\cellcolor[HTML]{9B9B9B}\specialcell{\textbf{Falhas de comunicação parciais: o efeito obtido é somente parte do}\\\textbf{efeito pretendido de acordo com a intenção do usuário}}} \\ \hline
\textbf{aspecto semiótico}  & \textbf{característica específica} & \textbf{etiqueta} \\ \hline
\multirow{-2}{*}{\specialcell{O usuário abandona\\uma semiose antes de\\obter o resultado espe-\\rado, e inicia outra\\com o mesmo propósito}}
    & \specialcell{porque, embora entenda a solução de IHC\\proposta, prefere seguir por outro caminho no\\momento} & \specialcell{Não,\\obrigado} \\ \cline{2-3} 
    &  porque não entende a solução de IHC proposta & \specialcell{Vai de\\outro\\jeito}  \\ \hline

    
\multicolumn{3}{||l||}{\cellcolor[HTML]{9B9B9B}\specialcell{\textbf{Falhas de comunicação temporárias: o efeito parcial do processo de}\\\textbf{interpretação (semiose) e de comunicação(interação) do usuário é}\\\textbf{inconsistente e incoerente com sua intenção de comunicação}}}   \\ \hline
\textbf{aspecto semiótico}  & \textbf{característica específica} & \textbf{etiqueta} \\ \hline

\multirow{3}{*}{\specialcell{O usuário interrompe\\temporariamente sua\\semiose}}
   &\specialcell{porque não encontra uma expressão\\apropriada para sua intenção de comunicação}& Cadê?\\ \cline{2-3} 
   &\specialcell{porque não percebe ou não entende a\\expressão do sistema (preposto do designer)}& \specialcell{Ué, o que\\houve?} \\ \cline{2-3} 
   &\specialcell{porque não consegue formular sua próxima\\intenção de comunicação}& E agora? \\ \hline
   
\multirow{3}{*}{\specialcell{O usuário percebe que\\seu ato comunicativo\\não foi bem-sucedido}}
   &\specialcell{porque percebeu que havia "falado" algo no\\contexto errado}&\specialcell{Onde\\estou?} \\ \cline{2-3} 
   & porque percebeu que havia "falado" algo errado & Epa!\\ \cline{2-3} 
   &\specialcell{porque não obteve o resultado esperado depois\\de conversar com o sistema (preposto do\\designer) por algum tempo, alternando vários\\turnos de fala com ele}&\specialcell{Assim não\\dá.}\\ \hline
   
\multirow{4}{*}{\specialcell{O usuário procura com-\\preender o ato comuni-\\cativo do sistema\\(preposto do designer)}}
   &através da metacomunicação implícita & \specialcell{O que é\\isto?}\\ \cline{2-3} 
   &através da metacomunicação explícita & Socorro!\\ \cline{2-3} 
   &\specialcell{testando várias hipóteses sobre o significado do\\que o sistema comunicou}&\specialcell{Por que\\não\\funciona?} \\ \hline   
\end{tabular}
\end{table}


%http://acervodigital.ufpr.br/bitstream/handle/1884/36666/Luiz%20Augusto%20Sakakibara.pdf?sequence=1


