\chapter{Projeto da Interface}

\section{Projeto}

Aplicação do método da avaliação da comunicabilidade

\section{Detalhes da Implementação}


\indent O sistema desenvolvido para este projeto é uma aplicação web chamada \textit{2Path}. O usuário deve se cadastrar no \textit{website} para ter acesso às redes metabólicas do banco de dados do sistema, bem como pesquisar por palavras chaves no mesmo. Nesta seção serão apresentadas as linguagens e ferramentas utilizadas no desenvolvimento do \textit{website}, as características, funcionalidades e limites do sistema e, por fim, as dificuldades enfrentadas na implementação do projeto.

\indent O sistema foi desenvolvido no amibiente de desenvolvimento integrado \textit{open source} Eclipse Java EE - \textit{Java Platform, Enterprise Edition}, versão Mars 4.5.2. Para simplificar a obtenção das dependências do projeto, ou seja, pacotes de arquivos java (extensão .jar), foi utilizada o Apache Maven, \textit{software} de gerenciamento de projeto e ferramenta de compreensão de programa. Este \textit{software} opera sobre o arquivo \textit{pom.xml}, onde POM significa \textit{Project Object Model} e contém as especificações de cada projeto que se tornará dependência do sistema em desenvolvimento, além de outros aspectos do código. O servidor selecionado para hospedagem local, \textit{localhost} porta 8080, do sistema foi o Apache TomCat versão 7.0.

\indent As páginas da aplicação foram desenvolvidas na linguagem de marcação XHTML, \textit{Extensible Hypertext Markup Language}, e a estilzação em CSS, \textit{Cascading Style Sheets}.

\indent 

JSF \\
PRIMEFACES


JAVASCRIPT \\
ANGULARJS \\
ORIENTBD

\section{Desafios}
 

O que foi o trabalho. 
Decrever todo o ambiente usado
Neste capítulo serão apresentados os primeiros resultados experimentais obtidos.