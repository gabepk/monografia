\chapter{Redes Metabólicas}


%% COPIADO DO ARTIGO DE REDES METABÓLICAS

%% -----------------------------------------------------------------
%% -----------------------------------------------------------------
%% -----------------------------------------------------------------

\indent A rede metabólica pode ser definida formalmente como uma coleção de objetos e as relações entre eles. Os objetos correspondem a compostos químicos, reações bioquímicas, enzimas e genes.

\indent compostos químicos, também chamados metabólitos, são pequenas moléculas que são importados / exportados e / ou sintetizadas / degradadas dentro de um organismo. Para a maioria dos metabolitos, a quantidade observada varia de acordo com o compartimento da célula e tecido no interior do qual o composto está presente. Os tecidos e células de facto conter um número de compartimentos líquidos separados uns dos outros por membranas de permeabilidade selectiva.

\indent Reações bioquímicas produzem um conjunto de um ou mais compostos (chamado de produtos) a partir de um outro conjunto de um ou mais compostos (chamados os substratos). Em teoria, uma reacção química pode ocorrer em ambos os sentidos. No entanto, sob determinadas condições fisiológicas, algumas reações ocorrem em apenas uma direção. Neste caso, eles são definidos como sendo irreversível, se todas as outras condições permanecem constantes. Dentro de uma célula, algumas reacções são espontâneas, mas a maioria são catalisada por uma ou várias enzimas que aceleram fortemente a sua velocidade. Uma enzima é uma proteína ou um complexo de proteína, codificada por um ou vários genes. Uma única enzima pode aceitar substratos distintas e pode catalisar a várias reacções, e, inversamente, uma única reacção pode ser catalisada por diversas enzimas. Elucidar as ligações entre genes, proteínas e reações (o chamado relacionamento GPR) não é uma tarefa trivial e é uma grande preocupação na reconstrução metabólica, como é discutido na próxima seção.


\indent A descoberta de enzimas por Eduard Buchner no início do século 20 separados o estudo das reacções químicas que compõem o metabolismo de um organismo a partir do estudo da biologia das suas células. Tais reacções são tradicionalmente agrupados em chamados vias metabólicas, o que pode por sua vez ser classificados como anabólico ou catabólico. Anabolismo, é a síntese de moléculas através do uso de energia e no consumo de agentes redutores (um agente de redução é uma substância que quimicamente reduz outras substâncias doando um ou vários electrões), enquanto o catabolismo corresponde à degradação de moléculas de rendimento de energia e a produção de redução agentes. As vias podem ser estudados, quer isoladamente, ou, uma vez que são sobrepostas, ser combinados em conjunto para produzir o que é referido como uma rede metabólica. Os benefícios de estudar toda a rede, em vez de percursos individuais são numerosas e incluem, por exemplo, a possibilidade de explorar alternativa vias.

%% -----------------------------------------------------------------
%% -----------------------------------------------------------------
%% -----------------------------------------------------------------


\cite{lacroixCTS08}