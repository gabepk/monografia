\chapter{2Path}

% =========================================================================================


% =========================================================================================

\section{Detalhes de Implementação}


\indent O sistema desenvolvido para este projeto é uma aplicação web chamada \textit{2Path}. O usuário deve se cadastrar no \textit{website} para ter acesso às redes metabólicas do banco de dados do sistema, bem como pesquisar por palavras chaves no mesmo. Nesta seção serão apresentadas as linguagens e ferramentas utilizadas no desenvolvimento do \textit{website}, as características, funcionalidades e limites do sistema e, por fim, as dificuldades enfrentadas na implementação do projeto.

\indent O sistema foi desenvolvido no amibiente de desenvolvimento integrado \textit{open source} Eclipse Java EE - \textit{Java Platform, Enterprise Edition}, versão Mars 4.5.2. Para simplificar a obtenção das dependências do projeto, ou seja, pacotes de arquivos java (extensão .jar), foi utilizada o Apache Maven\footnote{\textit{Software} de gerenciamento de projeto e ferramenta de compreensão de programa.}. Este \textit{software} opera sobre o arquivo \textit{pom.xml}, onde POM significa \textit{Project Object Model} e contém as especificações de cada projeto que se tornará dependência do sistema em desenvolvimento, além de outros aspectos do código. O servidor selecionado para hospedagem local, \textit{localhost} porta 8080, do sistema foi o Apache TomCat versão 7.0.

\indent As páginas da aplicação foram desenvolvidas na linguagem de marcação XHTML, \textit{Extensible Hypertext Markup Language}, e a estilzação em CSS, \textit{Cascading Style Sheets}. Para conexão entre \textit{front-end} e \textit{back-end} foram utilizados JSF\footnote{Especificação Java para criação de componentes de interface de aplicação \textit{web}.}, Primefaces\footnote{Biblioteca com uma coleção de componentes de interface voltadas para JSF.}, e AngularJS, para a construção do grafo de vias metabólicas.

\indent JSF: \url{https://projetos.inf.ufsc.br/arquivos_projetos/projeto_1214/Ferramenta\%20de\%20mapeamento\%20de\%20UIDs\%20para\%20JSF.pdf}

\indent CORES: \url{http://www.hermancerrato.com/graphic-design/images/color-images/the-meaning-of-colors-book.pdf}, \url{http://www.awwwards.com/trendy-web-color-palettes-and-material-design-color-schemes-tools.html}, \url{https://medium.com/wdstack/how-the-bootstrap-grid-really-works-471d7a089cfc#.k7w39z2uh}

% Communication websites which market to individual customers on a one-to-one basis would benefit with some blue in their marketing. Hi-tech and computer technology businesses can benefit from most shades of blue combined with gray.

\subsection{Banco de Dados OrientDB}

\indent O banco de dados escolhido para armazenar as enzimas, compostos, reações e demais elementos foi o OrientDB.\\
CITAR: (1) Graph database, (2) SQL, (3) Commercial Friendly License, (4) Low Total Cost of Ownership (TCO) Community Edition, (5) Open Source. \url{http://orientdb.com/why-orientdb/}, \url{http://orientdb.com/docs/last/Java-API.html}

\subsection{Desafios}
 
O que foi o trabalho. 
Decrever todo o ambiente usado
Neste capítulo serão apresentados os primeiros resultados experimentais obtidos.