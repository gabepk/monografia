\chapter{2Path: Aplicação Web}

\indent O sistema desenvolvido para este projeto é uma aplicação web chamada \textit{2Path}. O usuário deve se cadastrar no \textit{website} para ter acesso às redes metabólicas do banco de dados do sistema, bem como pesquisar por palavras chaves no mesmo. Neste caítulo serão apresentadas as linguagens e ferramentas utilizadas no desenvolvimento do \textit{website}, as características, funcionalidades e limItes do sistema e, por fim, as dificuldades enfrentadas na implementação do projeto.

\section{Implementação}

\indent O sistema foi desenvolvido no amibiente de desenvolvimento integrado \textit{open source} Eclipse Java EE - \textit{Java Platform, Enterprise Edition}, versão Mars 4.5.2. A plataforma Eclipse foi projeta com o objetivo de agilizar o desenvolvilmento de recursos integrados baseando-se em um modelo de \textit{plug-in}. Na \textit{workbench} no Eclipse, cada \textit{plug-in} é responsável por pequenas tarefas, tais como compilar, testar ou debugar \cite{eclipseDoc}.

\indent Para simplificar a obtenção das dependências do projeto, ou seja, pacotes de arquivos java (extensão .jar), foi utilizada o Apache Maven, \textit{software} de gerenciamento de projeto e ferramenta de compreensão de programa. Este \textit{software} opera sobre o arquivo \textit{pom.xml}, onde POM significa \textit{Project Object Model} e contém as especificações de cada projeto que se tornará dependência do sistema em desenvolvimento, além de outros aspectos do código. No exemplo abaixo, o fragmento do \textit{pom.xml} indica o \textit{groupID} - código único entre a organização ou projeto, \textit{artifactId} - nome do projeto, \textit{version} - versão do projeto que será baixada e \textit{scope} - escopo em que o projeto será necessário no sistema (compilação, execução ou teste).

\begin{lstlisting}[language=XML]
<dependencies>
(...)
	<!-- PrimeFaces (biblioteca de componentes) -->
	<dependency>
		<groupId>org.primefaces</groupId>
		<artifactId>primefaces</artifactId>
		<version>3.5</version>
		<scope>compile</scope>
	</dependency>
(...)
</dependencies>
\end{lstlisting}

\indent O servidor selecionado para <<<>>> o sistema na rede, \textit{localhost} porta 8080, foi o Apache TomCat versão 7.0. Este software é uma implementação \textit{open source} das quatro tecnologias \cite{oracle14} a seguir:
\begin{itemize}
	\item \textit{Java Servlet}: 
	\item \textit{JavaServer Pages}:
	\item \textit{Java Expression Language}:
	\item \textit{Java WebSocket}:
\end{itemize}

\indent COLOCAR IMAGEM DA ARQUITETURA MVC : An MVC EBookShop with Servlets, JSPs, and JavaBeans Deployed in Tomcat  \cite{chuan11} \\

\indent As quatro páginas da aplicação foram desenvolvidas na linguagem de marcação XHTML, \textit{Extensible Hypertext Markup Language}, e a estilzação em CSS, \textit{Cascading Style Sheets}. Com a primeira é possível criar objetos na página \textit{web} através de componentes nativos e não nativos da linguagem chamados \textit{tags}. As principais \textit{tags} são apresentadas na Tabela \ref{}. Já com CSS é possível customizar cada objeto da página web, alterando seu tamanho, posição, cor, fonte, e várias outras características. Para tal, o objeto por ser alterado individualmente através de seu ID; em conjunto, com objetos da mesma classe ou \textit{tag}.

\indent 

JSF \\
PRIMEFACES

\section{Visualização das redes metabólicas}

JAVASCRIPT \\
ANGULARJS \\
ORIENTBD

\section{Desafios}
 

O que foi o trabalho. 
Decrever todo o ambiente usado
Neste capítulo serão apresentados os primeiros resultados experimentais obtidos.
